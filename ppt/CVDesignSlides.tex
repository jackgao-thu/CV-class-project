%!TEX program = xelatex
\documentclass[aspectratio=169]{beamer}
\usepackage[UTF8]{ctex}
\usepackage{graphicx}
\usepackage{booktabs}
\usepackage{amsmath}
\usepackage{xcolor}
\usepackage{hyperref}

\usetheme{Madrid}
\setbeamertemplate{navigation symbols}{}
\setbeamertemplate{footline}[frame number]

\title{面向胸部CT报告生成的切片感知对齐与双路负向约束}
\author{高健 \and 赵越}
\institute{清华大学人工智能学院}
\date{\today}

\begin{document}

\begin{frame}
    \titlepage
\end{frame}

\begin{frame}{任务背景}
    \begin{itemize}
        \item 面向胸部三维CT的自动报告生成,需对跨切片病灶给出有依据的描述。
        \item 通用多模态大模型易走“文本捷径”,忽视稀疏的视觉线索,导致幻觉或模板化输出。
        \item 目标:提升模型对关键切片的依赖度,抑制无视觉证据的生成。
    \end{itemize}
\end{frame}

\begin{frame}{核心挑战}
    \begin{itemize}
        \item 图像线索稀疏:有效病灶切片比例低,跨切片关联弱。
        \item 文本先验过强:报告模板同质化,易被模型滥用。
        \item 视觉对齐困难:基座 ViT 在医学 CT 上经验不足,时间/空间对齐不稳。
    \end{itemize}
\end{frame}

\begin{frame}{方法概览}
    \begin{columns}[T]
    \begin{column}{0.55\linewidth}
        \begin{itemize}
            \item 基于 qwen2.5-VL 动态分辨率流水线,保持视觉-语言压缩与 MRoPE 时间对齐。
            \item 增加切片注意力池化与两类负向采样,显式强化视觉依赖。
            \item 以最小改动接入原对话模板,兼容现有训练/推理链路。
        \end{itemize}
    \end{column}
    \begin{column}{0.45\linewidth}
        \centering
        \includegraphics[width=\linewidth]{../计算机视觉课程大作业修改版/fig/method.pdf}
    \end{column}
    \end{columns}
\end{frame}

\begin{frame}{数据示例}
    \begin{columns}[T]
    \begin{column}{0.48\linewidth}
        \centering
        \includegraphics[width=\linewidth]{../计算机视觉课程大作业修改版/fig/chest_example_1.png}\\
        \small 病灶切片(存在结节/磨玻璃影)
    \end{column}
    \begin{column}{0.48\linewidth}
        \centering
        \includegraphics[width=\linewidth]{../计算机视觉课程大作业修改版/fig/chest_example_2.png}
    \end{column}
    \end{columns}
\end{frame}

\begin{frame}{对比现象}
    \centering
    \includegraphics[width=0.85\linewidth]{../计算机视觉课程大作业修改版/fig/compare.pdf}
    \vspace{4pt}
\end{frame}

\begin{frame}{模型架构}
    \centering
    \includegraphics[width=0.7\linewidth]{../计算机视觉课程大作业修改版/fig/qwen2.5vl_arc.jpeg}
    \vspace{4pt}
\end{frame}

\begin{frame}{双路负向采样}
    \begin{itemize}
        \item 噪声负样本(Ours-Gauss):构造全局高斯噪声体积 + 原报告,迫使模型在无图像证据时输出高困惑度。
        \item 异类负样本(Ours-Heter):替换为标签差异显著的 CT 体积 + 原报告,打破文本模板依赖。
        \item 训练时正负样本成对出现,保持其他字段一致以减少分布偏移。
    \end{itemize}
\end{frame}

\begin{frame}{损失与优化}
    \[
        \mathcal{L} = \mathbb{E}_{(x,y)\in \mathcal{B}_{\text{pos}}}\! \text{CE}(x,y) - \beta\, \mathbb{E}_{(x,y)\in \mathcal{B}_{\text{neg}}}\! \text{CE}(x,y),\quad \alpha{=}1,\ \beta{=}0.1
    \]
    \begin{itemize}
        \item 负样本项取负号,直接最大化其困惑度,抑制无依据的生成。
        \item 辅助策略:梯度裁剪 + 负损失上界,避免训练早期发散。
    \end{itemize}
\end{frame}

\begin{frame}{数据与预处理}
    \begin{itemize}
        \item 数据集:CT-RATE(25k+ 胸部 CT 序列,21k 患者),提供报告与 18 类病灶标签。
        \item 预处理:HU 转换;重采样至 $0.75 \times 0.75 \times 1.5$ mm;窗宽 [-1000, 1000] 并归一化到 [-1, 1]。
        \item 存储形态:完整 preprocessed 与 numpy 版 preprocessed\_npy 方便快速加载。
    \end{itemize}
\end{frame}

\begin{frame}{训练设置}
    \begin{itemize}
        \item 设备:2 张 A100,训练 1 epoch;学习率 $2\times10^{-5}$,batch size 2(梯度累积 8)。
        \item 与基座一致的视觉压缩与对话模板,仅调整负样本构造与损失。
        \item 验证集挑选超参,最佳 checkpoint 在测试集评估。
    \end{itemize}
\end{frame}

\begin{frame}{训练曲线}
    \centering
    \includegraphics[width=0.85\linewidth]{../计算机视觉课程大作业修改版/fig/ours_train_loss.png}
    \vspace{4pt}
    \small 正负样本交替训练,损失收敛平稳。
\end{frame}

\begin{frame}{主要结果}
    \centering
    \begin{tabular}{lccc}
        \toprule
        方法 & BLEU\_1 相对提升 & BLEU\_4 相对提升 & Coverage 提升 \\
        \midrule
        无负向约束(基础) & -- & -- & -- \\
        Ours-Gauss & +29.1\% & +24.3\% & +14.5\% \\
        Ours-Heter & +28.7\% & +23.5\% & +13.9\% \\
        \bottomrule
    \end{tabular}
    \vspace{6pt}\\
    \small 负向采样显著降低幻觉,提升临床相关性。
\end{frame}

\begin{frame}{局限与讨论}
    \begin{itemize}
        \item 正负样本需成对加载,训练吞吐降低。
        \item 负损失上界的手工设定可能限制进一步收益。
        \item 仍缺少对病灶局部解释性的显式约束。
    \end{itemize}
\end{frame}

\begin{frame}{未来工作}
    \begin{itemize}
        \item 引入切片级对比/检索,增强跨切片显式对齐。
        \item 结合局部病灶提示或掩码监督,提升小病灶敏感度。
        \item 探索自适应负样本权重调度,平衡语言流畅度与视觉依赖。
    \end{itemize}
\end{frame}

\end{document}
